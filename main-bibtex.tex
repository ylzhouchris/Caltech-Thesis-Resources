%%%%%%%%%%%% 
%% Please rename this main-bibtex.tex file and the output PDF to
%% [lastname_firstname_graduationyear]
%% before submission.
%%
%% This .tex file is for use with BibTeX. Please use
%% main.tex instead if you prefer BibLaTeX.
%%%%%%%%%%%%

\documentclass[12pt]{caltech_thesis}
\usepackage[hyphens]{url}
\usepackage{lipsum}
\usepackage{graphicx}

\usepackage{todonotes}

%% Tentative: newtx for better-looking Times
\usepackage[utf8]{inputenc}
\usepackage[T1]{fontenc}
\usepackage{newtxtext,newtxmath}

\usepackage[numbers,sort&compress]{natbib}
\usepackage{bibunits}
\defaultbibliographystyle{apalike} % reference style
\setcitestyle{authoryear,open={(},close={)}} % citestyle in text
\usepackage{enumitem}

\begin{document}


\maketitle
\title

\mainmatter

\chapter{Urban Growth}
\begin{bibunit}




\section{Urban Intensification}
\citep{Searle2011} Sydney

England, Compact City Policies \citep{Williams1999}
Back in the 2000s, urban intensification was seen in land use planning as a way of delivering sustainable development: it optimises the use of previously developed land, reduces travel demand and revitalises and repopulates cities. \citep{Williams1999}

However, by drawing on existing research and data some contradictions and complexities within both the content of intensification policies and their means of implementation are revealed. The paper reviews the origins of, and theory behind, compact city policies, and revealed the weaknesses in intensification policies. \citep{Williams1999}



\section{Drivers for Land Use Changes}
Land-use change is driven by synergetic factor combinations of resource scarcity leading to an increase in the pressure of production on resources, changing opportunities created by markets, outside policy intervention, loss of adaptive capacity, and changes in social organization and attitudes. \citep{Lambin2003}


\section{Projections of Urban Expansion}
\citep{Seto2012} to 2030

\cite{waddell2002urbansim} UrbanSim: Modeling urban development for land use, transportation, and environmental planning


\section{Environmental Impacts of Urban Growth}
\citep{Guneralp2008} Shenzhen, Global Env. Change

We identify three phases of urban growth and develop scenarios to evaluate the impact of urban growth on several environmental indicators: land use, air quality, and demand for water and energy. 

Demand for water and electricity will rise, and the city will become increasingly vulnerable to shortages of either. 



\subsection{Effects to Surrounding Areas}
However, the environmental impacts outside of Shenzhen may increase as demands for natural resources increase and Shenzhen pushes its manufacturing industries out of the municipality.\citep{Guneralp2008}


\section{Land Resources in Shenzhen}
A study in 2008 projected that all developable land in Shenzhen will be urban by 2020 and the increase in the number of vehicles will be a major source of air pollution.\citep{Guneralp2008}



\subsection{Strategic Spatial Planning}
\citep{Hersperger2018} Review Global Environment Change

To date land-change science has devoted little attention to spatial policy and planning in urban landscapes despite the widely accepted premise that planning affects urban land change. \citep{Hersperger2018}

This is primarily due to lack of relevant data and an underdeveloped theoretical understanding regarding the impact of spatial planning on urban land change. \citep{Hersperger2018}

Based on \textbf{a synthesis} of the current literature on how spatial planning is implemented in land-change models, and drawing from the literature on planning evaluation, we sketch a research agenda to further develop the understanding of these three components and their interconnections as well as their application in quantitative land-change modelling approaches for urban regions.\citep{Hersperger2018}




\renewcommand{\bibsection}{\section*{\refname}}
\putbib[ownpubs]  % the .bib file for this chapter
\end{bibunit}

\chapter{GeoDesign}
\begin{bibunit}

\section{Use of Science to Guide City Planning and Practice}
\citep{Sallis2016}
Land-use and transport policies contribute to worldwide epidemics of injuries and non-communicable diseases through traffic exposure, noise, air pollution, social isolation, low physical activity, and sedentary behaviours. Motorised transport is a major cause of the greenhouse gas emissions that are threatening human health. 

\textbf{Urban and transport planning and urban design policies in many cities do not reflect the accumulating evidence that, if policies would take health effects into account, they could benefit a wide range of common health problems.} \citep{Sallis2016}



Such an approach promises to be a powerful strategy for improvements in population health on a permanent basis.\citep{Sallis2016}






\citep{Readings2010}
Geographic information system (GIS) technology has a long history of driving environmental understanding and decision making. Policy makers, planners, scientists, and many others worldwide rely on GIS for data management and scientific analysis. As the challenges facing our natural and human environments evolve to new levels of complexity, our tools must also evolve. To this end, a dedicated group of people has been actively pursuing the development of GeoDesign theory, concepts, and tools. Design is about purpose and intentions; it's about seeing in our mind's eye what could be, then creating it. Based on concepts found in Ian McHarg's seminal Design With Nature, GeoDesign integrates geographic science with design, resulting in a systematic methodology for geographic planning and decision making. GeoDesign brings geographic analysis into any design process, resulting in designs that more closely follow natural systems. This benefits both people and 


\section{Recommended Urban Forms}
The primary recommendation of this paper is for cities to actively pursue \textbf{compact and mixed-use urban designs} that encourage a transport modal shift away from private motor vehicles towards walking, cycling, and public transport. \citep{Sallis2016}



\renewcommand{\bibsection}{\section*{\refname}}
\putbib[ownpubs]  % the .bib file for this chapter
\end{bibunit}



\chapter{Climate Change and Extreme Heat}

\section{Climate Projections}
\begin{bibunit}
\paragraph{\citep{Doan2018} IJCL FutureUrbanization}
Dynamical downscaling of three different Coupled Model Intercomparison Project Phase 5 (CMIP5) global climate models driven with two different representative concentration pathway (RCP) emission scenarios.

This study numerically evaluates the impacts of future urbanization and global climate change on the thermal environment of this city.

The Weather Research and Forecasting (WRF) model is used to produce these projections, having first been updated with current and future (master plan-based) land use data with a horizontal resolution of 1 km.

The results show that, in rural areas, the spatially averaged monthly mean air temperature in April is projected to increase by 1.2 and 1.7 °C by the 2050s under the RCP4.5 and RCP8.5 scenarios, respectively. 

In newly urbanized areas, an additional warming of 0.5 °C is expected under both scenarios, which corresponds to 20–30\% of the global warming. 

In particular, the additional warming due to urbanization can exceed 0.8 °C at night. 

\textbf{The impact of future urbanization (0.5 °C) is comparable to the difference in the temperature increases achieved under the different RCP scenarios.}


\renewcommand{\bibsection}{\section*{\refname}}
\putbib[ownpubs]  % the .bib file for this chapter
\end{bibunit}


\section{Extreme Heat Events and Impacts}
\begin{bibunit}
\subsection{Severe Heat Waves World-Wide}
France,2003 \citep{Vandentorren2006}

Canada \citep{Smoyer-Tomic2003}

US Government Resources, Weather Fatalities \citep{CenterforDiseaseControlandPrevention2008}


\subsection{Human Health Impacts}
People with pre-existing medical conditions were likely to be vulnerable during heat waves and need information on how to adjust daily routines to heat waves.\citep{Vandentorren2006}

\subsection{Call for Better Urban Planning}
The temperature around the building was a major risk factor. In the long term, building insulation and urban planning must be adapted to provide protection from possible heat waves. \citep{Vandentorren2006}

\renewcommand{\bibsection}{\section*{\refname}}
\putbib[ownpubs]  % the .bib file for this chapter
\end{bibunit}







\chapter{Climate Change Mitigation and Adaptation}




\section{Land Use Planning}
\begin{bibunit}

\subsection{Indications of Effectiveness}
\paragraph{\citep{Doan2018} IJCL}
This study numerically evaluates the impacts of future urbanization and global climate change on the thermal environment of this city.

The Weather Research and Forecasting (WRF) model is used to produce these projections, having first been updated with current and future (master plan-based) land use data with a horizontal resolution of 1 km.

In rural areas, the spatially averaged monthly mean air temperature in April is projected to increase by 1.2 and 1.7 °C by the 2050s under the RCP4.5 and RCP8.5 scenarios, respectively. 

In newly urbanized areas, an additional warming of 0.5 °C is expected under both scenarios, which corresponds to 20–30\% of the global warming. 

In particular, the additional warming due to urbanization can exceed 0.8 °C at night. 


\subsection{}

\citep{Xu2019} EI TBD

\citep{Huang2014}
sustainable land use planning
+growth modeling


\renewcommand{\bibsection}{\section*{\refname}}
\putbib[ownpubs]  % the .bib file for this chapter
\end{bibunit}


\chapter{Spatial Optimization}
\section{Reviews}
\begin{bibunit}
\citep{Tong2012}

\citep{Cao2019}

\section{Optimzation Algorithms}
Most engineering optimization algorithms are based on numerical linear and nonlinear programming methods that require substantial gradient information and usually seek to improve the solution in the neighborhood of a starting point. These algorithms, however, reveal a limited approach to complicated real-world optimization problems. If there is more than one local optimum in the problem, the result may depend on the selection of an initial point, and the obtained optimal solution may not necessarily be the global optimum. \citep{lee2005new}



From the previous literature,several solutions have been used for solving different types of spatial forest resource planning problems, among which exact solutions
include Metropolis algorithm (Van Deusen, 1999), mixed
integer programming (Bare and Eldon, 1969; Bevers and Hof, 1999;
McDill and Rebain, 2003), and dynamic programming (Hoganson
and Borges, 1998); while metaheuristic approaches include penalty
function with simulated annealing (Lockwood and Moore,
1993), tabu search (Murray and Church, 1999), and evolutionary
algorithm (Liu et al., 2006), among others. Some articles focused on
conducting comprehensive experimental comparison on several
heuristics, e.g., see (Liu et al., 2006; Pukkala and Kurttila, 2005).


GRadient Descent

In the past, optimization problems were usually handled by
gradient search methods, which require substantial gradient information.
However, lots of practical optimization problems are
much complex and may have multimodal solution space, so that
the gradient search may be unstable and difficult (Lee and Geem,
2005). Hence, most recent works focused on developing metaheuristic
algorithms for optimization problems. This work proposes
a metaheuristic algorithm based on CA for the problem of concern.




Evolutionary Algorithm
The evolutionary algorithm (EA) is a stochastic global search
method, among which genetic algorithm (GA) is the most popular
type of EA, and has been proved to be successful for a variety of
environmental optimization problems, e.g., water distribution
system (Bi et al., 2015; Zheng et al., 2015; Gupta et al., 1999),
forecasting value of agricultural imports (Lee and Liu, 2014), estimation
of soil organic and mineral fraction densities (Crowe et al.,
2006), among others. A survey on the applications of EA in water
resources can be found in (Maier et al., 2014). EA allows the solution
representation to be a sequence of base-10 numbers, tables
or other data structures. It works with a population of individuals
(candidate solutions) and tries to optimize the answer by using
three basic evolutionary principles, including selection, crossover
(also called recombination), and mutation. The pseudo code of EA
for the forest resource planning problem of concern in (Liu et al.,
2006) is stated in Algorithm 1, in which the iteration number N
is used to count the total iterations of the main loop (see Lines 3, 9,
and 10).


- GA

-- selection operators
    roulette-wheel selection (Fogel, 1995),




- Cultural Algorithm (Reynolds, 1994)


Cultural algorithm (CA) proposed by Reynolds (1994) is a class of
evolutionary algorithm (Fogel, 1995) based on some theories from
sociology and archaeology that try to formulate cultural evolution.
From those theories, the cultural evolution can be thought of as an
inheritance process according to the genetic material that an
offspring inherits from its parents, and the knowledge acquired by
individuals through generations that guides behavior of the individuals.
The concept of CA has already found applications in a
variety of areas, such as timetabling problem (Soza et al., 2011),
economic load dispatch problem (dos Santos Coelho et al., 2009),
multi-machine power system stabilizer (Khodabakhshian and
Hemmati, 2013), extracting urban occupational plans (Jayyousi
and Reynolds, 2014), among others.

This work proposes a cultural algorithm approach to the above
spatial forest resource planning problem. The cultural algorithm
(CA) (Reynolds, 1994) is an evolutionary algorithm (Fogel, 1995; Yu
and Gen, 2010) which improves performance of evolutionary
search by extracting domain knowledge of the problem of concern
during the search process. In addition to the conventional evolutionary
settings that act on a population of individuals (candidate
solutions) and intend to iteratively improve fitness (probably in
terms of the objective function of the problem of concern) of each
individual, it maintains a belief space consisting of a half of individuals
with better fitness values from the current population,
as well as a leader to guide the whole population. During the
search process, the belief space and the leader are updated by
incorporation of the extracted problem-specific knowledge, and
they influence each individual in the population to obtain better
solutions. As a result, this work investigates how to design a CA
specifically for the spatial forest resource planning problem with
the above concerns. For performance evaluation, the proposed
algorithm is experimentally compared with the previous bestknown
simulated annealing approach to the same problem in
(Liu et al., 2006).


Simulated Anealing and Linear Programming \citep{tarp1997spatial}

PSO \citep{ma2011land}

\section{SO Methods}

\citep{ward2003integrating} 
a SDSS tool for addressing sustainable urban change
evaluating urban change integrating cellular automata

SDSS represents an approach for advancing spatial analysis or model functionality for an application specific planning problem (Densham 1994; Malczewski 1999; Church et al. 2000). SDSS has traditionally served as a means for extending the services provided by GIS to support decision making processes.

\citep{stewart2004genetic} A genetic approach

\citep{huang2008seeking} pareto front

\citep{Cao2011} Extension to NSGA-II
\citep{cao2012sustainable} boundary-based fast genetic algorithm

\citep{stewart2014multiobjective}This component uses a simple representation of the proximity of related land uses to each other as a function of distances between parcel centroids.

\citep{eikelboom2015spatial} SO for geodesign


\citep{cao2013coarse} vector based land use optimization, coarse-grained parallel genetic algorithm

\cite{liu2015land} Game Theory, a land-use spatial optimization model that emphasizes coordinating local land-use competitions is constructed

\citep{Cao2019a} Time Steps. How to reach the optimal land use plan over time
\citep{bi2019life}spatial-temporal coverage, A genetic algorithm is applied to optimize the rollout of DWPT infrastructure both spatially and temporally in order to minimize life cycle costs, GHG, and energy burdens:

\citep{zhu2019effects} effects of spatial configuration in spatial optimization. A framework. 


\section{SO Applications}
\subsection{sustainable land use planning}
\citep{ligmann2008spatial} a generative technique for sustainable multiobjective land use allocation

\citep{parish2012multimetric} Win-Win. optimize landscape design of cellulosic bioenergy crop plantings may simultaneously improve water quality and increase profits for farmer‐producers while achieving a feedstock‐production goal. 

\citep{cao20171} sustainable land use planning
\citep{yao2019evaluation} sustainable land use plan, evaluation
\citep{yao2019evaluation} fire locations, visualization



\subsection{Other Applications}
\citep{hof1992spatial} for forest management
This paper presents several nonlinear formulations for land allocation that optimize spatial layout for a single time period and that have the property that the number of choice variables increases linearly with the level of spatial resolution.
Objectives: the amount of edge, the juxtaposition of different habitat types for cover versus feeding needs, the dispersal distance between favorable habitats, and the minimum size of a patch of habitat. 

\citep{nevo1996spatial} spatial optimization of wildlife habitat
A decision support system was developed for the design of optimal distributions of cover types that provide prescribed levels of wildlife habitat. The objective of the optimization is to minimize the cost of modifying existing cover types. The system was implemented on a UNIX workstation and includes a graphical user interface utilizing X Windows and Motif. Linked to the system is a geographic information system (GIS) that supplies a non-linear mathematical programming model with spatial data and stores the optimization output for further processing and display.

\citep{tarp1997spatial} forest management, simulated annealing and linear programming. (1) the incremental cost (determined by use of the LP model) of an optimal adjacency model solution, and (2) the potential damage cost resulting from adjacency characteristics such as windthrow and bark injuries.


\citep{church2000mapping} map evacuation risk on a transportation network, spatial continguity constraints

\citep{farhan2008siting} supporting transit planning, for siting park-and-ride facilities
spatial objectives:covering as much potential demand as possible, locating park-and-ride facilities as close as possible to major roadways, and siting such facilities in the context of an existing system.

\citep{christensen2009spatial} protected area placement incorporating ecological, social and economical criteria

\cite{delmelle2012identifying} +spatial interaction modeling. simulated annealing, identify bus stop redundancies

\citep{gaddis2014spatial} attain water quality targets
\citep{leuthold2012large} so for European electricity market
In this paper, we present a large-scale spatial model of the European electricity market including both generation and the physical transmission network (DC Load Flow approach). 

The model was developed to analyze various questions on market design, congestion management, and investment decisions, with a focus on Germany and Continental Europe. 

It is a bottom-up model combining electrical engineering and economics: its objective function is welfare maximization, subject to line flow, energy balance, and generation constraints. 

The model provides simulations on an hourly basis, taking into account variable demand, wind input, unit commitment, start-up costs, pump storage, and other details. Various forms of spatial price discrimination can be implemented, such as locational marginal pricing (“nodal pricing”), or zonal pricing. With over 2,000 nodes and over 3,000 lines, this is one of the largest models developed to date, and allows a highly differentiated spatial analysis. We report modeling results regarding efficient congestion management for Germany and Europe, optimal network expansion under the aspect of increased wind energy production, and the impact of network constraints on location decisions of generation investments.

\citep{liu2015land} forest management using cultural algorithm, good evaluation indicators. 


\renewcommand{\bibsection}{\section*{\refname}}
\putbib[ownpubs]  % the .bib file for this chapter
\end{bibunit}

\section{Typical applications}
\begin{bibunit}





\renewcommand{\bibsection}{\section*{\refname}}
\putbib[ownpubs]  % the .bib file for this chapter
\end{bibunit}






\chapter{Model Evaluation}
Sensitivity Analysis
This kind of analysis is conceived as a stage in the model evaluation that examines the extent of output variation of a model when parameters are systematically varied over a range of interest.



\chapter{This is the Seventh Chapter}
\chapter{This is the Eighth Chapter}

\bibliographystyle{plainnat}
\bibliography{example}

\appendix

\chapter{Questionnaire}
\chapter{Consent Form}

\printindex

\theendnotes

%% Pocket materials at the VERY END of thesis
\pocketmaterial
\extrachapter{Pocket Material: Map of Case Study Solar Systems}


\end{document}
